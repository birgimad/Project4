\subsection{Improved Monto Carlo Method}
\label{subsec:ImprovedMCMethod}
In the improved brute force Monto Carlo method the coordinates of the function to be integrated are changed from cartesian coordinates to spherical coordinates with $r_i$ being exponentially distributed on the interval $[0;\infty)$, $\theta_i$ being uniformly distributed on the interval $[0;\pi]$, and $\phi_i$ being uniformly distributed on the interval $[0,2\pi]$. 
That means that the PDF's of $\theta_i$ and $\phi_i$ are given by
\begin{align}
	p(\theta_i ) = \frac{1}{\pi} \qquad \text{and} \qquad p(\phi_i) = \frac{1}{2\pi}
	\label{sec:ImprovedMCMethod1}
\end{align} 
whilst the PDF for the radial component is given by
\begin{align}
	p(r_i) = exp(-r_i )
	\label{sec:ImprovedMCMethod2}
\end{align}
Since the considered function in spherical coordinates is given by \matref{eq:CartesianSpherical3},
it is an advantage to do a change of the radial variables into
\begin{align}
	u_i = 2\alpha r_i
\end{align}  
which is also exponentially distributed in the interval $[0;\infty)$ with the PDF fiven by \matref{sec:ImprovedMCMethod2}.
This change in variables could of cause also have been applied in the Laguerre method described in \secref{sec:LaguerreMethod}.
This gives
\begin{align}
	du_i = 2\alpha dr_i
\end{align}
and hence the function that is to be evaluated becomes
\begin{align}  
   f(u_1, u_2, \theta_1, \theta_2, \phi_1, \phi_2 ) = 
   \frac{1}{(2\alpha)^5}
   \frac{e^{-u_1 - u_2 }}{\sqrt{u_1^2 + u_2 ^2 -2 u_1 u_2 cos(\beta)}}  
u_1 ^2 u_2 ^2 sin(\theta_1 ) sin(\theta_2 )    
   \label{eq:ImprovedMCMethod4}
\end{align}
When dividing this function by the PDF's for the radial components $u_1$ ans $u_2$, it is evident that the exponential $e^{-u_1 - u_2 }$ will disappear.
That means that when evaluating the integral with the Monte Carlo method, it becomes
\begin{align}
	I = \frac{1}{N} \sum_{i=1} ^N \frac{\tilde{f} (u_i, \tilde{u}_i, \theta_i, \tilde{\theta}_i, \phi_i, \tilde{\phi}_i )}{p(\theta_i)p(\tilde{\theta}_i) p(\phi_i) p(\tilde{\phi}_i)} 
	= \frac{1}{N} \sum_{i=1} ^N \frac{\tilde{f} (u_i, \tilde{u}_i, \theta_i, \tilde{\theta}_i, \phi_i, \tilde{\phi}_i )}{4\pi ^4}
\end{align}
in which $\tilde{f} = f/e^{-u_1 - u_2 }$.

The source code below shows the for loop for the improved Monte Carlo method can be seen below. 
The difference between this and the for loop for the brute force Monte Carlo method given in \subsecref{subsec:MCMethod}, is the definition of the random variables and explicit statement of the total PDF $px$.
\begin{lstlisting}
for (int i=0;i<N;i++){
	for (int j=0;j<6;j++){
		x[j] = ((double) rand() / (RAND_MAX)); 
		//random numbers generated in the interval(0,1)
    }
		//making the random numbers for u_i, theta_i and phi_i on respective interval.            
            double u1 = -log(1-x[0]);
            double u2 = -log(1-x[1]);
            double theta1 = x[2]*pi;
            double theta2 = x[3]*pi;
            double phi1 = x[4]*2*pi;
            double phi2 = x[5]*2*pi;
		//evaluating the function in the random u_i, theta_i and phi_i
            fx = func(u1,u2,theta1,theta2,phi1,phi2);
		//computing the product of the PDF's of theta_i and phi_i
            px = 1/(4*pi*pi*pi*pi);
		//computing the parts to be summed up
            Montec += fx/px;
            Montesqr += (fx/px)*(fx/px);
}
		//computing integral value and variance from which the standard deviation is calculated
       double integral_value = Montec/((double) N );
       double Montesqr1 = Montesqr/((double) N);
       double variance = Montesqr1-integral_value*integral_value;
       double standard_deviation = sqrt(variance/ ((double) N));
\end{lstlisting}
The function \textit{func} that is called in the for loop is in spherical coordinates and with the change of variable $u_i = 2\alpha r_i$ as described in this section, unlike the function called in the for loop for the brute force Monte Carlo method, which is in cartesian coordinates. 
The random radial number $u_i$ is computed from the uniformly distributed random number $x_j$ in $[0;1]$ by the formula
\begin{align}
	u_i = -ln(1-x_j)
	\label{eq:ImprovedMCMethod6}
\end{align}
since the probability has to be conserved, yielding
\begin{align}
	p(u) du = exp(-u) du = dx = p(x)dx
	\label{eq:ImprovedMCMethod5}
\end{align}
in which $p(x) = 1$ since $x$ is uniformly distributed in $[0;1]$.
Integration of \matref{eq:ImprovedMCMethod5} from $0$ to $u$ yields
\begin{align}
	x = 1-exp(-u)
\end{align}
from which \matref{eq:ImprovedMCMethod6} follows.