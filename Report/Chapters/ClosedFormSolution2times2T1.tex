\subsection{Closed Form Solutions for the $2\times 2$ case with $T=1.0$}
\label{sec:ClosedFormSolution2times2T1}
\fxnote{introduce value of partition function}
In this project, the temperature is given in the units of $[k_B T/J] = [1/\beta J]$.
To distinguish this from the temperature $T$ in the ordinary unit of Kelvin \fxnote{ok??}, the considered temperature is, in this section, written as $\tilde{T}$.
In \fxnote{ref, to section for 2x2 case}, the c++ code for computing the expectation value of the energy and the magnetization and the specific heat and susceptibility of the $2\times 2$ spin case with temperature $\tilde{T} = 1.0$ is introduced.
The section is dedicated to find the closed form solutions for these quantities for this situation with the purpose of testing the code introduced in the mentioned section. \fxnote{any ideas to improving this section??}

With $\tilde{T} = 1/\beta J = 1.0$, the partition function in \matref{eq:PartitionFunction2times2} gives the value
\begin{align}
	Z = 12+4\cosh(8) \approx 6000
\end{align}
\fxnote{write more exact result for Z}
The expectation value of the magnetization is surely still zero, whilst the expectation value of the absolute value of the magnetization becomes
\begin{align}
	\left< \mathcal{M} \right> = \frac{1}{Z} ( 8 e^8 +4) \approx 3.9926
\end{align}
With the temperature given in units of $[k_B T/J]$, the formulas for determining the expectation value of the energy and the specific heat and susceptibility, given in the previous section, must be slightly modified to give a value.
E.g. the expectation value of the energy given by \matref{eq:ExpectationEnergy2times2} will have to be divided by $J$, giving the computed expectation value in the unit of $[E/J]$. 
For a temperature of $\tilde{T} = 1.0$, the computed expectation value of the energy is then
\begin{align}
	\tilde{\left< E \right>} 
	= \frac{\left< E \right>}{J}	
	= \frac{16}{Z} (e^{-8}-e^{8}) \approx -7.9839 
\end{align} 
This is approximately the same as the lowest energy state of the system (see \tabref{tab:ClosedFormSolution1}), which is also what was to be expected for low temperatures. 
To find $C_v$ and $\chi$, \matref{eq:SpecificHeat2times2} and \eqref{sec:Susceptibility2times2} are considered, and with a temperature $\tilde{T}  = 1.0$, they give the values
\begin{align*}
	\tilde{C}_v = \frac{C_v}{J}
	\approx 0.12830
	\qquad \text{and} \qquad
	\tilde{\chi} = \chi J \approx 0.03209
\end{align*}
\fxnote{check results, especially $\chi$}
The values of the quantities gained in this section for the $2\times 2$ spin case with $\tilde{T} = 1.0$ in units of $[k_B T/J] = [1/\beta J]$ are collected in the table below.
\begin{table}[H]
\centering
\caption{Various quantity values for the $2\times 2$ spin case with temperature $\tilde{T} = 1.0$ in units of $[k_B T/J] = [1/\beta J]$.}
\begin{center}
\begin{tabular}{ | c | c | c | c | c | }
  \hline			
  $\left< \tilde{E} \right>$ & $\left< \mathcal{M} \right> $ &  $\left< |\mathcal{M} | \right> $ & $\tilde{C}_v $ & $\tilde{\chi}$  \\
  \hline
  $-7.9839$ & $0$ & $3.9926$ & $0.12830$ & $0.03209$ \\
  \hline
\end{tabular}
\end{center}
\label{tab:ClosedFormSolution2times2T1}
\end{table}
\fxnote{fix $\chi$}